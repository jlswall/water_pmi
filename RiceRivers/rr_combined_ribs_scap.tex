% $Header: /cvsroot/latex-beamer/latex-beamer/solutions/generic-talks/generic-ornate-15min-45min.en.tex,v 1.5 2007/01/28 20:48:23 tantau Exp $

\documentclass{beamer}

\usepackage{caption}
\captionsetup{labelformat=empty,labelsep=none,font=scriptsize}
\setlength{\abovecaptionskip}{0pt}

\usepackage{color}
%% These definitions are based on darkred at
%% http://www.december.com/html/spec/colorcmyk.html
\definecolor{darkred}{cmyk}{0, 1, 1, 0.45}
\newcommand{\jul}{\textcolor{darkred}}
\newcommand{\jan}{\textcolor{blue}}

% This file is a solution template for:

% - Giving a talk on some subject.
% - The talk is between 15min and 45min long.
% - Style is ornate.



% Copyright 2004 by Till Tantau <tantau@users.sourceforge.net>.
%
% In principle, this file can be redistributed and/or modified under
% the terms of the GNU Public License, version 2.
%
% However, this file is supposed to be a template to be modified
% for your own needs. For this reason, if you use this file as a
% template and not specifically distribute it as part of a another
% package/program, I grant the extra permission to freely copy and
% modify this file as you see fit and even to delete this copyright
% notice. 


\mode<presentation>
{
  \usetheme{Warsaw}
  % or ...

  \setbeamercovered{transparent}
  % or whatever (possibly just delete it)
}


\usepackage[english]{babel}
% or whatever

\usepackage[latin1]{inputenc}
% or whatever

\usepackage{times}
\usepackage[T1]{fontenc}
% Or whatever. Note that the encoding and the font should match. If T1
% does not look nice, try deleting the line with the fontenc.


%% \title[Short Paper Title] % (optional, use only with long paper titles)
%% {Presentation Title}
\title[]{Models combining rib and scapula samples - Rice Rivers study}
%\subtitle {Eastern CASTNET sites, May-Sep.~2001} % (optional)

%% \author[Author, Another] % (optional, use only with lots of authors)
%% {F.~Author\inst{1} \and S.~Another\inst{2}}
%% % - Use the \inst{?} command only if the authors have different
%% %   affiliation.
%% \author[Swall et al.]{Jenise Swall\inst{1}, Ana Rappold\inst{2}, and Lucas Neas\inst{2}
% - Use the \inst{?} command only if the authors have different
%   affiliation.

%% \institute[Universities of Somewhere and Elsewhere] % (optional, but mostly needed)
%% {
%%   \inst{1}%
%%   Department of Computer Science\\
%%   University of Somewhere
%%   \and
%%   \inst{2}%
%%   Department of Theoretical Philosophy\\
%%   University of Elsewhere}
%% % - Use the \inst command only if there are several affiliations.
%% % - Keep it simple, no one is interested in your street address.
 %% \institute[VCU]
 %% {
 %%   \inst{1}%
 %%   Dept.\ of Statistical Sciences and Operations Research\\
 %%   Virginia Commonwealth University
 %%   \and
 %%   \inst{2}%
 %%   National Health and Environmental Effects Research Laboratory\\
 %%   U.S.~Environmental Protection Agency
 %% }

%% \date[Short Occasion] % (optional)
%% {Date / Occasion}
\date{July 2022}

%% \subject{Talks}
% This is only inserted into the PDF information catalog. Can be left
% out. 



% If you have a file called "university-logo-filename.xxx", where xxx
% is a graphic format that can be processed by latex or pdflatex,
% resp., then you can add a logo as follows:

% \pgfdeclareimage[height=0.5cm]{university-logo}{university-logo-filename}
% \logo{\pgfuseimage{university-logo}}



% Delete this, if you do not want the table of contents to pop up at
% the beginning of each subsection:
\AtBeginSection[]
{
  \begin{frame}<beamer>{Outline}
    \tableofcontents[currentsection,currentsubsection]
  \end{frame}
}


% If you wish to uncover everything in a step-wise fashion, uncomment
% the following command: 

%\beamerdefaultoverlayspecification{<+->}

\useoutertheme{infolines}

\begin{document}

\begin{frame}
   \titlepage
\end{frame}

%% \begin{frame}{Outline}
%%  \tableofcontents
  % You might wish to add the option [pausesections]
%% \end{frame}


% Since this a solution template for a generic talk, very little can
% be said about how it should be structured. However, the talk length
% of between 15min and 45min and the theme suggest that you stick to
% the following rules:  

% - Exactly two or three sections (other than the summary).
% - At *most* three subsections per section.
% - Talk about 30s to 2min per frame. So there should be between about
%   15 and 30 frames, all told.


%% %%%%%%%%%%%%%%%%%%%%%%%%%%%%%%%%%%%%%%%%%%%%%%%%%%%%%%%%%%



%% %%%%%%%%%%%%%%%%%%%%%%%%%%%%%%%%%%%%%%%%%%%%%%%%%%%%%%%%%%
\section{Preliminaries}

\begin{frame}{Combining rib and scapula}

  \begin{itemize}
    \item I re-ran the random forest models using ribs and scapula samples combined.
    \item The models don't distinguish between ribs and scapula, so neither type
   is given more weight than the other.
    \item In general, we expect improved overall model performance due to the
    larger sample sizes obtained by combining ribs and scapulae samples.  This
    is particularly relevant for the models using swabs, since the sample sizes
    are smaller than for the bone data.
    \item However, if there are notable differences in the rib and scapula
    samples, this could adversely affect model fit and interpretation.
  \end{itemize}

\end{frame}



\begin{frame}{Choosing taxa to include}
  
  \begin{itemize}
    \item In our previous separate analyses for ribs and scapulae, we have
    required that a taxon must have a prevalence of at least 1\%  in at least 2
    samples in order to be included in the analysis.
    \item When doing the analysis using combined rib and scapula samples, we
    require that the taxa must have met this standard for \textbf{both} ribs and
    scapulae to be included.  This means that the number of taxa considered in
    the model is smaller than the numbers used in the individual rib and scapula
    models.
  \end{itemize}

\end{frame}
%% %%%%%%%%%%%%%%%%%%%%%%%%%%%%%%%%%%%%%%%%%%%%%%%%%%%%%%%%%%



%% %%%%%%%%%%%%%%%%%%%%%%%%%%%%%%%%%%%%%%%%%%%%%%%%%%%%%%%%%%
\section{Using swab samples}


% %%%%%%%%%%%%%%%%%%%%%%%%%%%%%%%%%%%
\subsection{Omitiing baseline observations (ADD 0)}

\begin{frame}{Random forest model combining rib and scapula samples (omitting ADD 0)}

  \begin{tabular}{lrrrr}
    Type & \# samples & \# taxa & RMSE & Expl.\ variation\\ \hline
    Ribs & 15 & 49 & 642.7 &      73.1\% \\
    Scapulae & 14 & 53 & 800.9 &  56.6\% \\
    Combined & 29 & 38 & 621.1 &  74.4\%
  \end{tabular}
  
  \vspace{0.1in}

\end{frame}


\begin{frame}{Pred.\ vs. actual ADD for swabs (without baseline obs.)}

  \begin{center}
    \begin{figure}
      \includegraphics[height=3.1in]
        {w_swabs/bacteria/use_families/rr_combined_family_no_baseline_predicted_vs_actual_ADD}
    \end{figure}
  \end{center}

\end{frame}



\begin{frame}{Influential taxa for swabs (omitting ADD 0)}

  \begin{center}
    \begin{figure}
      \includegraphics[height=2.85in]
        {w_swabs/bacteria/use_families/rr_combined_family_no_baseline_6panels}
    \end{figure}
  \end{center}

\end{frame}


\begin{frame}{Influential taxa are different}
  
  \begin{itemize}
    \item The influential taxa are different for the analyses using ribs only,
    scapulae only, and combined ribs and scapulae.
    \item This is partly because, to be included in the combined rib/scapula
    model, a taxon must have a prevalence of at least 1\% in at least 2 rib
    samples \textbf{and} at least 2 swab samples.
    \begin{itemize}
      \item The following influential taxa in the rib model were not included in
    the combined rib/scapula model:\\
    Haliangiaceae and Rhodospirillaceae
    \item The following influential taxa in the scapula model were not included
    in the combined rib/scapula model:\\
    Methylocystaceae and Tissierellaceae
    \end{itemize}
    \item These following taxa were in the "top 10" influential taxa for all
    three models:\\
    Enterobacteriaceae and Methylococcaceae
  \end{itemize}

\end{frame}
% %%%%%%%%%%%%%%%%%%%%%%%%%%%%%%%%%%%



% %%%%%%%%%%%%%%%%%%%%%%%%%%%%%%%%%%%
\subsection{Including baseline observations (with ADD 0)}

\begin{frame}{Random forest model combining rib and scapula samples (using ADD 0)}

  \begin{tabular}{lrrrr}
    Type & \# samples & \# taxa & RMSE & Expl.\ variation\\ \hline
    Ribs & 18 & 49 & 607.8 & 81.8\% \\
    Scapulae & 17 & 53 & 681.4 & 76.6\% \\
    Combined & 35 & 38 & 580.2 & 83.2\%
  \end{tabular}
  
  \vspace{0.1in}

\end{frame}


\begin{frame}{Pred.\ vs. actual ADD for swabs (with baseline obs.)}

  \begin{center}
    \begin{figure}
      \includegraphics[height=3.1in]
        {w_swabs/bacteria/use_families/rr_combined_family_w_baseline_predicted_vs_actual_ADD}
    \end{figure}
  \end{center}

\end{frame}



\begin{frame}{Influential taxa for swabs (with ADD 0)}

  \begin{center}
    \begin{figure}
      \includegraphics[height=2.85in]
        {w_swabs/bacteria/use_families/rr_combined_family_w_baseline_6panels}
    \end{figure}
  \end{center}

\end{frame}


\begin{frame}{Influential taxa are different}
  %% Change this!
  \begin{itemize}
    \item Methylococcaceae is influential in each of the analyses.
    \item The following influential taxa in the rib model were not included in
    the combined rib/scapula model:\\
    Rhodospirillaceae and Desulfobacteraceae
    \item The following influential taxon in the scapula model was not included
    in the combined rib/scapula model:\\
    Methylocystaceae
    \item The following taxa were in the "top 10" influential taxa for all
    three models:\\
    Methylococcaceae, Crenotrichaceae, Holophagaceae
  \end{itemize}

\end{frame}

\end{document}




\subsection[Rib swabs]{Working with swabs from ribs}


\begin{frame}{Implementing the random forest model for ribs (omitting baseline obs)}

  \noindent Full model omitting baseline observations (not using ADD 0):
  \begin{itemize}
    \item Utilized 49 family-level taxa.
    \item RMSE: $\approx$ 642.7  
    \item Explained variation: $\approx$ 73.1\%
  \end{itemize}
  % RMSE went down from about 670 and expl. variation up from about 72 after
  % dataset updated.
  \vspace{0.1in}

  \noindent We have 15 observations for the rib swabs when we exclude the
  baseline observations (as compared with 13 before the dataset was updated).
  % Compared with 13 observations before dataset updated.

\end{frame}



\begin{frame}{Implementing the random forest model for ribs (using baseline obs)}

  \noindent Full model using baseline observations (using ADD 0):
  \begin{itemize}
    \item Utilized 49 family-level taxa.
    \item RMSE: $\approx$ 600.6  
    \item Explained variation: $\approx$ 82.2\%
  \end{itemize}
  
  \vspace{0.1in}

  \noindent We have 18 observations for the rib swabs when we include the
  baseline observations.

\end{frame}
% %%%%%%%%%%%%%%%%%%%%%%%%%%%%%




% %%%%%%%%%%%%%%%%%%%%%%%%%%%%%
\section[Scapula swabs]{Working with swabs from scapulae}

\begin{frame}{Implementing the random forest model for scapulae (omitting baseline obs)}

  \noindent Full model omitting baseline observations (not using ADD 0):
  \begin{itemize}
    \item The model utilized 53 family-level taxa. % same as before update
    \item RMSE: $\approx$ 800.9
    \item Explained variation: $\approx$ 56.6\%
  \end{itemize}
  % RMSE is up from 771.2.  Expl. variation is down from 59.8.
  \vspace{0.1in}

  \noindent We have 14 observations for the scapula swabs when baseline
  observations are excluded.
  % Also had 14 observations before dataset was updated.

\end{frame}


\begin{frame}{Implementing the random forest model for scapulae (using baseline obs)}

  \noindent Full model omitting baseline observations (using ADD 0):
  \begin{itemize}
    \item The model utilized 53 family-level taxa. 
    \item RMSE: $\approx$ 681.4
    \item Explained variation: $\approx$ 76.6\%
  \end{itemize}
  \vspace{0.1in}

  \noindent We have 17 observations for the scapula swabs when we include the
  baseline observations.

\end{frame}
% %%%%%%%%%%%%%%%%%%%%%%%%%%%%%



% %%%%%%%%%%%%%%%%%%%%%%%%%%%%%
\section[Swabs from ribs and scapulae]{Comparing rib and scapula models (swabs)}


% %%%%%
\subsection[No baseline]{Omitting baseline observations}

\begin{frame}{Influential taxa for swabs (without baseline obs.)}

  \begin{center}
    \begin{figure}
      \includegraphics[width=4.25in]
        {use_families/rr_swabs_rib_scapula_family_no_baseline_4panels}
    \end{figure}
  \end{center}

\end{frame}


\begin{frame}{Pred.\ vs. actual ADD for swabs (without baseline obs.)}

  \begin{center}
    \begin{figure}
      \includegraphics[width=4.75in]
        {use_families/rr_swabs_rib_scapula_family_no_baseline_predicted_vs_actual_ADD}
    \end{figure}
  \end{center}

\end{frame}
% %%%%%


% %%%%%
\subsection[With baseline]{Using baseline observations}

\begin{frame}{Influential taxa for swabs (with baseline obs.)}

  \begin{center}
    \begin{figure}
      \includegraphics[width=4.25in]
        {use_families/rr_swabs_rib_scapula_family_w_baseline_4panels}
    \end{figure}
  \end{center}

\end{frame}


\begin{frame}{Pred.\ vs. actual ADD for swabs (with baseline obs.)}

  \begin{center}
    \begin{figure}
      \includegraphics[width=4.75in]
        {use_families/rr_swabs_rib_scapula_family_w_baseline_predicted_vs_actual_ADD}
    \end{figure}
  \end{center}

\end{frame}
% %%%%%

% %%%%%%%%%%%%%%%%%%%%%%%%%%%%%



% %%%%%%%%%%%%%%%%%%%%%%%%%%%%%
\section[Bones vs.\ swabs]{Comparing models using bones vs.\ swabs}

% %%%%%%%%%%
% %%%%%
\subsection[Ribs]{Using ribs}


% %%%%%
\subsubsection[No baseline]{Omitting baseline observations}

\begin{frame}{Ribs: Influential taxa (no baseline obs.)}

  \begin{center}
    \begin{figure}
      \includegraphics[width=4.25in]
        {../../rr_rib_bones_vs_swabs_no_baseline_4panels}
    \end{figure}
  \end{center}

\end{frame}


\begin{frame}{Ribs: Pred.\ vs. actual ADD (no baseline obs.)}

  \begin{center}
    \begin{figure}
      \includegraphics[width=4.75in]
        {../../rr_rib_bones_vs_swabs_no_baseline_predicted_vs_actual_ADD}
    \end{figure}
  \end{center}

\end{frame}
% %%%%%


% %%%%%
\subsubsection[With baseline]{Using baseline observations}

\begin{frame}{Ribs: Influential taxa (with baseline obs.)}

  \begin{center}
    \begin{figure}
      \includegraphics[width=4.25in]
        {../../rr_rib_bones_vs_swabs_w_baseline_4panels}
    \end{figure}
  \end{center}

\end{frame}


\begin{frame}{Ribs: Pred.\ vs. actual ADD (with baseline obs.)}

  \begin{center}
    \begin{figure}
      \includegraphics[width=4.75in]
        {../../rr_rib_bones_vs_swabs_w_baseline_predicted_vs_actual_ADD}
    \end{figure}
  \end{center}

\end{frame}
% %%%%%
% %%%%%%%%%%



% %%%%%%%%%%
% %%%%%%%%%%
% %%%%%
\subsection[Scapulae]{Using scapulae}

% %%%%%
\subsubsection[No baseline]{Omitting baseline observations}

\begin{frame}{Scapulae: Influential taxa (no baseline obs.)}

  \begin{center}
    \begin{figure}
      \includegraphics[width=4.25in]
        {../../rr_scapula_bones_vs_swabs_no_baseline_4panels}
    \end{figure}
  \end{center}

\end{frame}


\begin{frame}{Scapulae: Pred.\ vs. actual ADD (no baseline obs.)}

  \begin{center}
    \begin{figure}
      \includegraphics[width=4.75in]
        {../../rr_scapula_bones_vs_swabs_no_baseline_predicted_vs_actual_ADD}
    \end{figure}
  \end{center}

\end{frame}
% %%%%%


% %%%%%
\subsubsection[With baseline]{Using baseline observations}

\begin{frame}{Scapulae: Influential taxa (with baseline obs.)}

  \begin{center}
    \begin{figure}
      \includegraphics[width=4.25in]
        {../../rr_scapula_bones_vs_swabs_w_baseline_4panels}
    \end{figure}
  \end{center}

\end{frame}


\begin{frame}{Scapulae: Pred.\ vs. actual ADD (with baseline obs.)}

  \begin{center}
    \begin{figure}
      \includegraphics[width=4.75in]
        {../../rr_scapula_bones_vs_swabs_w_baseline_predicted_vs_actual_ADD}
    \end{figure}
  \end{center}

\end{frame}
% %%%%%



% %%%%%%%%%%
% %%%%%%%%%%%%%%%%%%%%%%%%%%%%%

\end{document}
