% $Header: /cvsroot/latex-beamer/latex-beamer/solutions/generic-talks/generic-ornate-15min-45min.en.tex,v 1.5 2007/01/28 20:48:23 tantau Exp $

\documentclass{beamer}

\usepackage{caption}
\captionsetup{labelformat=empty,labelsep=none,font=scriptsize}
\setlength{\abovecaptionskip}{0pt}

\usepackage{color}
%% These definitions are based on darkred at
%% http://www.december.com/html/spec/colorcmyk.html
\definecolor{darkred}{cmyk}{0, 1, 1, 0.45}
\newcommand{\jul}{\textcolor{darkred}}
\newcommand{\jan}{\textcolor{blue}}

% This file is a solution template for:

% - Giving a talk on some subject.
% - The talk is between 15min and 45min long.
% - Style is ornate.



% Copyright 2004 by Till Tantau <tantau@users.sourceforge.net>.
%
% In principle, this file can be redistributed and/or modified under
% the terms of the GNU Public License, version 2.
%
% However, this file is supposed to be a template to be modified
% for your own needs. For this reason, if you use this file as a
% template and not specifically distribute it as part of a another
% package/program, I grant the extra permission to freely copy and
% modify this file as you see fit and even to delete this copyright
% notice. 


\mode<presentation>
{
  \usetheme{Warsaw}
  % or ...

  \setbeamercovered{transparent}
  % or whatever (possibly just delete it)
}


\usepackage[english]{babel}
% or whatever

\usepackage[latin1]{inputenc}
% or whatever

\usepackage{times}
\usepackage[T1]{fontenc}
% Or whatever. Note that the encoding and the font should match. If T1
% does not look nice, try deleting the line with the fontenc.


%% \title[Short Paper Title] % (optional, use only with long paper titles)
%% {Presentation Title}
\title[]{Models combining rib and scapula samples - Henley Lake study}
%\subtitle {Eastern CASTNET sites, May-Sep.~2001} % (optional)

%% \author[Author, Another] % (optional, use only with lots of authors)
%% {F.~Author\inst{1} \and S.~Another\inst{2}}
%% % - Use the \inst{?} command only if the authors have different
%% %   affiliation.
%% \author[Swall et al.]{Jenise Swall\inst{1}, Ana Rappold\inst{2}, and Lucas Neas\inst{2}
% - Use the \inst{?} command only if the authors have different
%   affiliation.

%% \institute[Universities of Somewhere and Elsewhere] % (optional, but mostly needed)
%% {
%%   \inst{1}%
%%   Department of Computer Science\\
%%   University of Somewhere
%%   \and
%%   \inst{2}%
%%   Department of Theoretical Philosophy\\
%%   University of Elsewhere}
%% % - Use the \inst command only if there are several affiliations.
%% % - Keep it simple, no one is interested in your street address.
 %% \institute[VCU]
 %% {
 %%   \inst{1}%
 %%   Dept.\ of Statistical Sciences and Operations Research\\
 %%   Virginia Commonwealth University
 %%   \and
 %%   \inst{2}%
 %%   National Health and Environmental Effects Research Laboratory\\
 %%   U.S.~Environmental Protection Agency
 %% }

%% \date[Short Occasion] % (optional)
%% {Date / Occasion}
\date{July 2022}

%% \subject{Talks}
% This is only inserted into the PDF information catalog. Can be left
% out. 



% If you have a file called "university-logo-filename.xxx", where xxx
% is a graphic format that can be processed by latex or pdflatex,
% resp., then you can add a logo as follows:

% \pgfdeclareimage[height=0.5cm]{university-logo}{university-logo-filename}
% \logo{\pgfuseimage{university-logo}}



% Delete this, if you do not want the table of contents to pop up at
% the beginning of each subsection:
% \AtBeginSection[]
% {
%  \begin{frame}<beamer>{Outline}
%    \tableofcontents[currentsection,currentsubsection]
%  \end{frame}
% }


% If you wish to uncover everything in a step-wise fashion, uncomment
% the following command: 

%\beamerdefaultoverlayspecification{<+->}

\useoutertheme{infolines}

\begin{document}

\begin{frame}
   \titlepage
\end{frame}

%% \begin{frame}{Outline}
%%  \tableofcontents
  % You might wish to add the option [pausesections]
%% \end{frame}


% Since this a solution template for a generic talk, very little can
% be said about how it should be structured. However, the talk length
% of between 15min and 45min and the theme suggest that you stick to
% the following rules:  

% - Exactly two or three sections (other than the summary).
% - At *most* three subsections per section.
% - Talk about 30s to 2min per frame. So there should be between about
%   15 and 30 frames, all told.


%% %%%%%%%%%%%%%%%%%%%%%%%%%%%%%%%%%%%%%%%%%%%%%%%%%%%%%%%%%%






%% %%%%%%%%%%%%%%%%%%%%%%%%%%%%%%%%%%%%%%%%%%%%%%%%%%%%%%%%%%
\section{Using swabs}


% %%%%%%%%%%%%%%%%%%%%%%%%%%%%%%%%%%%
\subsection{Omiting baseline observations}

\begin{frame}{Comparison of rib, scapula, and combined models (omitting ADD 0)}

  \begin{tabular}{lrrrr}
    Type & \# samples & \# taxa & RMSE & Expl.\ variation\\ \hline
    Ribs & 27 & 21 & 609.6 & 77.9\% \\
    Scapulae & 26 & 37 & 526.5 & 82.4\% \\
    Combined & 53 & 17 & 607.8 & 77.3\%
  \end{tabular}
  
  \vspace{0.1in}

\end{frame}


% \begin{frame}{Pred.\ vs. actual ADD for swabs (without baseline obs.)}

%   \begin{center}
%     \begin{figure}
%       \includegraphics[height=3.1in]
%         {w_swabs/bacteria/use_families/hl_combined_family_no_baseline_predicted_vs_actual_ADD}
%     \end{figure}
%   \end{center}

% \end{frame}



% \begin{frame}{Influential taxa for swabs (omitting ADD 0)}

%   \begin{center}
%     \begin{figure}
%       \includegraphics[height=2.85in]
%         {w_swabs/bacteria/use_families/hl_combined_family_no_baseline_6panels}
%     \end{figure}
%   \end{center}

% \end{frame}


% \begin{frame}{Influential taxa are different}
  
%   \begin{itemize}
%     \item The influential taxa are different for the analyses using ribs only,
%     scapulae only, and combined ribs and scapulae.
%     \item This is partly because, to be included in the combined rib/scapula
%     model, a taxon must have a prevalence of at least 1\% in at least 2 rib
%     samples \textbf{and} at least 2 swab samples.
%     \begin{itemize}
%       \item The following influential taxa in the rib model were not included in
%     the combined rib/scapula model:\\
%     Haliangiaceae and Rhodospirillaceae
%     \item The following influential taxa in the scapula model were not included
%     in the combined rib/scapula model:\\
%     Methylocystaceae and Tissierellaceae
%     \end{itemize}
%     \item The following taxa were in the "top 10" influential taxa for all
%     three models:\\
%     Enterobacteriaceae and Methylococcaceae
%   \end{itemize}

% \end{frame}
% %%%%%%%%%%%%%%%%%%%%%%%%%%%%%%%%%%%



% %%%%%%%%%%%%%%%%%%%%%%%%%%%%%%%%%%%
\subsection{Including baseline observations (with ADD 0)}

\begin{frame}{Comparison of rib, scapula, and combined models (using ADD 0)}

  \begin{tabular}{lrrrr}
    Type & \# samples & \# taxa & RMSE & Expl.\ variation\\ \hline
    Ribs & 30 & 21 & 582.3 & 83.6\% \\
    Scapulae & 28 & 37 & 588.2 & 81.3\% \\
    Combined & 58 & 17 & 623.2 & 80.2\%
  \end{tabular}

\end{frame}



%% %%%%%%%%%%%%%%%%%%%%%%%%%%%%%%%%%%%%%%%%%%%%%%%%%%%%%%%%%%





%% %%%%%%%%%%%%%%%%%%%%%%%%%%%%%%%%%%%%%%%%%%%%%%%%%%%%%%%%%%
\section{Using bones}


% %%%%%%%%%%%%%%%%%%%%%%%%%%%%%%%%%%%
\subsection{Omiting baseline observations}

\begin{frame}{Comparison of rib, scapula, and combined models (omitting ADD 0)}

  \begin{tabular}{lrrrr}
    Type & \# samples & \# taxa & RMSE & Expl.\ variation\\ \hline
    Ribs &  &  &  & \% \\
    Scapulae & &  &  &  \% \\
    Combined & 164 & 23 & 541.0 & 87.4\%
  \end{tabular}

  \vspace{0.2in}



\end{frame}




% %%%%%%%%%%%%%%%%%%%%%%%%%%%%%%%%%%%




% %%%%%%%%%%%%%%%%%%%%%%%%%%%%%%%%%%%
\subsection{Including baseline observations (with ADD 0)}

\begin{frame}{Comparison of rib, scapula, and combined models (using ADD 0)}

  % \begin{tabular}{lrrrr}
  %   Type & \# samples & \# taxa & RMSE & Expl.\ variation\\ \hline
  %   Ribs & 84 & 41 & 473.3 & 93.7\% \\
  %   Scapulae & 115 & 55 & 498.5 & 93.4\% \\
  %   Combined & 199 & 38 & 560.9 & 91.4\%
  % \end{tabular}
  
  \vspace{0.2in}



\end{frame}




\end{document}
