% $Header: /cvsroot/latex-beamer/latex-beamer/solutions/generic-talks/generic-ornate-15min-45min.en.tex,v 1.5 2007/01/28 20:48:23 tantau Exp $

\documentclass{beamer}

\usepackage{caption}
\captionsetup{labelformat=empty,labelsep=none,font=scriptsize}
\setlength{\abovecaptionskip}{0pt}

\usepackage{color}
%% These definitions are based on darkred at
%% http://www.december.com/html/spec/colorcmyk.html
\definecolor{darkred}{cmyk}{0, 1, 1, 0.45}
\newcommand{\jul}{\textcolor{darkred}}
\newcommand{\jan}{\textcolor{blue}}

% This file is a solution template for:

% - Giving a talk on some subject.
% - The talk is between 15min and 45min long.
% - Style is ornate.



% Copyright 2004 by Till Tantau <tantau@users.sourceforge.net>.
%
% In principle, this file can be redistributed and/or modified under
% the terms of the GNU Public License, version 2.
%
% However, this file is supposed to be a template to be modified
% for your own needs. For this reason, if you use this file as a
% template and not specifically distribute it as part of a another
% package/program, I grant the extra permission to freely copy and
% modify this file as you see fit and even to delete this copyright
% notice. 


\mode<presentation>
{
  \usetheme{Warsaw}
  % or ...

  \setbeamercovered{transparent}
  % or whatever (possibly just delete it)
}


\usepackage[english]{babel}
% or whatever

\usepackage[latin1]{inputenc}
% or whatever

\usepackage{times}
\usepackage[T1]{fontenc}
% Or whatever. Note that the encoding and the font should match. If T1
% does not look nice, try deleting the line with the fontenc.


%% \title[Short Paper Title] % (optional, use only with long paper titles)
%% {Presentation Title}
%% \title[]{Initial findings}
%\subtitle {Eastern CASTNET sites, May-Sep.~2001} % (optional)

%% \author[Author, Another] % (optional, use only with lots of authors)
%% {F.~Author\inst{1} \and S.~Another\inst{2}}
%% % - Use the \inst{?} command only if the authors have different
%% %   affiliation.
%% \author[Swall et al.]{Jenise Swall\inst{1}, Ana Rappold\inst{2}, and Lucas Neas\inst{2}
% - Use the \inst{?} command only if the authors have different
%   affiliation.

%% \institute[Universities of Somewhere and Elsewhere] % (optional, but mostly needed)
%% {
%%   \inst{1}%
%%   Department of Computer Science\\
%%   University of Somewhere
%%   \and
%%   \inst{2}%
%%   Department of Theoretical Philosophy\\
%%   University of Elsewhere}
%% % - Use the \inst command only if there are several affiliations.
%% % - Keep it simple, no one is interested in your street address.
 %% \institute[VCU]
 %% {
 %%   \inst{1}%
 %%   Dept.\ of Statistical Sciences and Operations Research\\
 %%   Virginia Commonwealth University
 %%   \and
 %%   \inst{2}%
 %%   National Health and Environmental Effects Research Laboratory\\
 %%   U.S.~Environmental Protection Agency
 %% }

%% \date[Short Occasion] % (optional)
%% {Date / Occasion}
%% \date{Oct.\ 2017}

%% \subject{Talks}
% This is only inserted into the PDF information catalog. Can be left
% out. 



% If you have a file called "university-logo-filename.xxx", where xxx
% is a graphic format that can be processed by latex or pdflatex,
% resp., then you can add a logo as follows:

% \pgfdeclareimage[height=0.5cm]{university-logo}{university-logo-filename}
% \logo{\pgfuseimage{university-logo}}



% Delete this, if you do not want the table of contents to pop up at
% the beginning of each subsection:
\AtBeginSection[]
{
  \begin{frame}<beamer>{Outline}
    \tableofcontents[currentsection,currentsubsection]
  \end{frame}
}


% If you wish to uncover everything in a step-wise fashion, uncomment
% the following command: 

%\beamerdefaultoverlayspecification{<+->}

\useoutertheme{infolines}

\begin{document}

%% \begin{frame}
%%   \titlepage
%% \end{frame}

\begin{frame}{Outline}
  \tableofcontents
  % You might wish to add the option [pausesections]
\end{frame}


% Since this a solution template for a generic talk, very little can
% be said about how it should be structured. However, the talk length
% of between 15min and 45min and the theme suggest that you stick to
% the following rules:  

% - Exactly two or three sections (other than the summary).
% - At *most* three subsections per section.
% - Talk about 30s to 2min per frame. So there should be between about
%   15 and 30 frames, all told.


%% %%%%%%%%%%%%%%%%%%%%%%%%%%%%%%%%%%%%%%%%%%%%%%%%%%%%%%%%%%



%% %%%%%%%%%%%%%%%%%%%%%%%%%%%%%
%% Introductory material
%% \section[Background]{Background ideas and info}
\section[Preparation]{Preparing the data}


\begin{frame}{Initial looks into family taxa}

  \noindent Data was read from Excel files:\\
  \texttt{HenleyLake\_Taxonomy.xlsx} - taxonomy info\\
  \texttt{HenleyLake\_SampleInformation.xlsx} - sample info (degdays, type, date)

  \vspace{0.1in}

  \noindent Family data from \texttt{Taxlevel 5-Family} worksheet
  \begin{itemize}
  \item Ribs: 154 family-level taxa represented
  \item Scapulae: 210 family-level taxa represented
  \item Water: 216 family-level taxa represented
  \end{itemize}
 
\end{frame}



\begin{frame}{Percentages unclassifed at the family level}

  \begin{center}
    \begin{figure}
      \includegraphics[width=3.25in]{family_perc_classif_by_add_type}
    \end{figure}
  \end{center}
  \vspace{-0.1in}
  {\footnotesize
  \begin{itemize}
    \item Percentages classified to family level: 90.5\% (ribs), 88.9\%
    (scapulae), 77.7\% (water).
    \item For ADD above 3000, rib samples had a high percentage of
  counts that could only be classified at the kingdom level.
  \end{itemize}
  }
\end{frame}



\begin{frame}{Dealing with "rare" taxa}

  \noindent To be included in the family-level analysis, we require that a taxon
  be identified at the family level and that it make up more than 1\% of the
  total counts for at least 2 samples.

  \begin{itemize}
    \item Ribs: 24 taxa
    \item Scapulae: 34 taxa
    \item Water: 31 taxa
  \end{itemize}

\end{frame}


\end{document}
